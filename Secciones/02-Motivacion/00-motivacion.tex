% Sección II
\section{Motivación}\label{sec:Motivacion}

La preocupación frente a las amenazas cibernéticas ha ido en incremento durante la
última década, impulsada por la creciente adopción de tecnologías digitales y su
aprovechamiento en diversos contextos organizacionales. Este escenario ha posicionado
a la ciberseguridad como un componente fundamental en los entornos tecnológicos
actuales, especialmente en aquellos donde la información constituye un activo
estratégico para la operación institucional. De acuerdo con Lallie et al.%~\cite{Lallie2025},
la dependencia cada vez mayor de la tecnología en múltiples ámbitos laborales ha
ampliado la superficie de ataque de las organizaciones, incrementando los riesgos
asociados a la integridad, confidencialidad y disponibilidad de la información.
Esta realidad impacta de manera directa a las instituciones de educación superior,
las cuales requieren infraestructuras tecnológicas capaces de soportar y proteger
adecuadamente sus activos de información.%~\cite{EDUCAUSE2023}

En el contexto de la Universidad del Quindío, si bien se dispone de controles de
seguridad y de una infraestructura tecnológica que soporta los servicios
institucionales, el grupo de investigación GRID administra activos de información
estratégicos, tales como resultados de investigación, servicios académicos y recursos
digitales, que demandan un nivel de protección ajustado a sus intereses actuales. En este
sentido, se identifica la oportunidad de apoyar el fortalecimiento de la arquitectura de seguridad del
entorno en la nube mediante la incorporación de un dispositivo UTM, la mejora en la
aplicación de controles de seguridad, y una monitorización mas detellada de lo que actualmente existe,
enfocados en las necesidades específicas del grupo de investigación%~\cite{EduTech2024}.

Bajo este escenario, el presente trabajo se motiva en la necesidad de optimizar y
robustecer la seguridad existente mediante la especificación de una arquitectura de
seguridad para la infraestructura de computación en la nube del grupo de investigación
GRID, alineada con los controles pertinentes de las normas ISO/IEC 27001, 27017 y 27018.
Dicha alineación busca mejorar la gestión de la seguridad de la información, garantizar
una protección más efectiva de los activos del grupo y contribuir al fortalecimiento
de la ciberseguridad institucional desde una perspectiva normativa y contextualizada.
