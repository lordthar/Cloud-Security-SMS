\documentclass[journal]{IEEEtran}

% Paquete para soporte del idioma español
% Importante para que la compilación de términos generados automaticamente se haga en español.
\usepackage[spanish]{babel}
% Use natbib for compressed citations
\usepackage[numbers,compress]{natbib}
\usepackage[utf8]{inputenc}
\usepackage{array}
\usepackage{booktabs}
\usepackage{xcolor}

%Paquetes para hacer cálculos.
\usepackage{calc}
\usepackage{xfp} % For more precise floating point calculations


\usepackage{multirow, array} % para las tablas
\usepackage{graphicx}
\usepackage{enumitem}

\usepackage{tabularx}

\usepackage{titlesec}
% Force line breaks after subsubsections
\titleformat{\subsubsection}
{\normalfont\normalsize\itshape}{\thesubsubsection}{1em}{}[\vspace{1ex}]
\titlespacing{\subsubsection}
{0pt} % left indentation
{3ex plus 1ex minus 0.5ex} % space before de subsubsection
{1ex plus 0.5ex minus 0.2ex} % space after de subsubsection
% ------ explaining the latter part: -----------
% 2ex = Base spacing (2 times the height of letter "x" in current font)
% plus 0.5ex = Maximum additional space LaTeX can add if needed
% minus 0.2ex = Maximum space LaTeX can remove if needed



% --- PAQUETE PARA HIPERVÍNCULOS ---
% La finalidad de su uso en este proyecto es otorgar enlaces a las citas hechas a traves del documento.
\usepackage{hyperref} % <---- Load this package LAST!
\hypersetup{
    colorlinks=true,
    linkcolor=blue,    % Color for internal links (e.g., sections)
    citecolor=blue,    % Color for citations
    filecolor=blue,    % Color for file links
    urlcolor=blue,     % Color for external URLs
}
% ---------------------------------

% Redefinir el nombre de las palabras clave a español
\renewcommand{\IEEEkeywordsname}{Palabras Clave}

% Redefinir el nombre de las tablas al español - CORRECT WAY with babel
\addto\captionsspanish{\renewcommand{\tablename}{Tabla}}

% Comando artesanal para escribir texto con dos propiedades
% italica y negrilla.
% uso -> \bolditalic{su texto aquí!!!!!!!}
\newcommand{\bolditalic}[1]{\textbf{\textit{#1}}}

% Estructura del documento como referencia para el futuro
%\part           Optional - for very large documents
%  \chapter      Top-level division
%    \section
%      \subsection
%        \subsubsection
%          \paragraph
%            \subparagraph


% !!! Importante
% Para la nomemclatura de archivos
% 1) Aquellos archivos (no incluye directorios) que empiezan con "00" deben ser considerados como
%    el archivo "main" de esa sección

\begin{document}

    \title{
        Seguridad en la nube\\
        Estudio de mapeo sistematico
    }

    \author{
        Miguel Angel Garcia Osorio,~\IEEEmembership{Estudiante de pregrado, Universidad del Quindío}, \\
        Huendy Caicedo Tangarife,~\IEEEmembership{Estudiante de pregrado, Universidad del Quindío} \\
        Luis Eduardo Sepúlveda Rodríguez,~\IEEEmembership{PhD, Universidad del Quindío} y \\
        Christian Andres Candela Uribe,~\IEEEmembership{Dr, Universidad del Quindío}
    }

% The paper headers
    %!TODO Colocar el nombre de la revista una vez lo sepamos.
    \markboth{Revista LA REVISTA,~Vol.~XX, No.~YY, Abril~2026}{Header de la derecha}


    \maketitle

    \begin{abstract}
        This document describes the most common article elements and how to use the IEEEtran class with \LaTeX \ to produce files that are suitable for submission to the IEEE.  IEEEtran can produce conference, journal, and technical note (correspondence) papers with a suitable choice of class options.
    \end{abstract}

    \begin{IEEEkeywords}
        Article submission, IEEE, IEEEtran, journal, \LaTeX, paper, template, typesetting.
    \end{IEEEkeywords}

%Seccion introducción
    %sección I
\section{Introducción}
\IEEEPARstart{E}{l} objetivo de la computación científica es la resolución
de problemas. La computadora resulta necesaria para este propósito debido a
que algunos problemas del mundo real frecuentemente presentan un nivel de dificultad
o complejidad que excede las capacidades de analítica o resolución humana, sin embargo,
estos pueden ser abordados efectivamente mediante el uso
de recursos computacionales.


No obstante, no todos los problemas científicos son manejables para una sola computadora. Existen problemas cuya ejecución en una sola maquina resulta
inviable debido a factores como su naturaleza o a el tamaño de su conjunto de datos. Es por esto
que los investigadores usan herramientas de computación distribuida para producir resultados. A su vez, la computación distribuida es una disciplina que también
ha encontrado interés en el ámbito educativo y tiene como
propósito el maximizar la cantidad de resultados producidos durante un periodo
largo de tiempo.

En este contexto surge HTCondor, un sistema creado por la
Universidad Wisconsin–Madison, especializado en la gestión de cargas
de trabajo y diseñado específicamente para tareas de cómputo intensivo.
HTCondor permite a los usuarios enviar tareas computacionales a un clúster,
donde el sistema gestiona de forma autónoma la asignación, planificación y distribución del
trabajo entre los nodos disponibles. El mecanismo de planificación opera bajo un
modelo de políticas bidireccional: tanto los propietarios de los recursos computacionales
como los usuarios solicitantes pueden establecer criterios y preferencias
que determinan dónde y bajo qué condiciones se ejecutarán las tareas.
Dichos trabajos computacionales vienen en la forma de lenguajes de programación
o \textit{contextos de ejecución} los cuales HTCondor llama
\textit{universos}. Hasta la fecha en la que se escribe el presente artículo
los universos HTCondor disponibles son los siguientes: \textit{vanilla, grid, java, scheduler,
    local, parallel, vm, container y docker}.


La diversidad de universos disponibles refleja la amplia gama de aplicaciones
que HTCondor puede soportar, desde computación tradicional hasta entornos
virtualizados y contenedorizados. No obstante, la literatura científica carece
de una clasificación sistemática que permita comprender cómo estos universos
se aplican en diferentes contextos y cuál es su impacto. En consecuencia,
el presente documento exponemos un estudio de mapeo sistemático que busca,
en primer lugar, clasificar trabajos relacionados con diversos dominios tecnológicos
como lo son la computación distribuída y paralela, el desarrollo de software,
virtualización, contenerización, la redes de computadora, entre otros. En segundo lugar,
se busca identificar y categorizar trabajos vinculados con los universos de
HTCondor como herramienta para fortalecer funciones esenciales universitarias
como: investigación, docencia, extensión e industria.

El resto del documento se estructura de la siguiente manera:
la Sección~\ref{sec:Motivacion} indica la motivación para este trabajo.
La Sección~\ref{sec:Trabajos-Relacionados} presenta trabajos relacionados.
%La Sección~\ref{sec:metodo-revision} describe el método utilizado para llevar a cabo el SMS.
%La Sección~\ref{sec:analisis-discusion} contiene el análisis y discusión del trabajo realizado.
%La Sección~\ref{sec:amenazas-validez} discute las amenazas a la validez, y finalmente,
%la Sección~\ref{sec:conclusiones} presenta las conclusiones.

%Sección motivación
    % Sección II
\section{Motivación}\label{sec:Motivacion}

La preocupación frente a las amenazas cibernéticas ha ido en incremento durante la
última década, impulsada por la creciente adopción de tecnologías digitales y su
aprovechamiento en diversos contextos organizacionales. Este escenario ha posicionado
a la ciberseguridad como un componente fundamental en los entornos tecnológicos
actuales, especialmente en aquellos donde la información constituye un activo
estratégico para la operación institucional. De acuerdo con Lallie et al.%~\cite{Lallie2025},
la dependencia cada vez mayor de la tecnología en múltiples ámbitos laborales ha
ampliado la superficie de ataque de las organizaciones, incrementando los riesgos
asociados a la integridad, confidencialidad y disponibilidad de la información.
Esta realidad impacta de manera directa a las instituciones de educación superior,
las cuales requieren infraestructuras tecnológicas capaces de soportar y proteger
adecuadamente sus activos de información.%~\cite{EDUCAUSE2023}

En el contexto de la Universidad del Quindío, si bien se dispone de controles de
seguridad y de una infraestructura tecnológica que soporta los servicios
institucionales, el grupo de investigación GRID administra activos de información
estratégicos, tales como resultados de investigación, servicios académicos y recursos
digitales, que demandan un nivel de protección ajustado a sus intereses actuales. En este
sentido, se identifica la oportunidad de apoyar el fortalecimiento de la arquitectura de seguridad del
entorno en la nube mediante la incorporación de un dispositivo UTM, la mejora en la
aplicación de controles de seguridad, y una monitorización mas detellada de lo que actualmente existe,
enfocados en las necesidades específicas del grupo de investigación%~\cite{EduTech2024}.

Bajo este escenario, el presente trabajo se motiva en la necesidad de optimizar y
robustecer la seguridad existente mediante la especificación de una arquitectura de
seguridad para la infraestructura de computación en la nube del grupo de investigación
GRID, alineada con los controles pertinentes de las normas ISO/IEC 27001, 27017 y 27018.
Dicha alineación busca mejorar la gestión de la seguridad de la información, garantizar
una protección más efectiva de los activos del grupo y contribuir al fortalecimiento
de la ciberseguridad institucional desde una perspectiva normativa y contextualizada.


%Sección trabajos relacionados
    %sección III
\section{Trabajos Relacionados}\label{sec:Trabajos-Relacionados}

No se identificaron estudios previos que compartan los objetivos de
esta investigación. Adicionalmente, basandose en los resultados obtenidos
en el presente trabajo, se evidenció que la literatura carece de trabajos
que examinen específicamente los universos de HTCondor, confirmando
la necesidad del presente mapeo sistemático.

Sin embargo, se encontraron los siguientes trabajos relacionados:

\begin{itemize}[label=\textbf{--}]
	\item Erickson et al. en el 2018 describen la importancia de tecnologías como la computación
    de alto rendimiento (HPC) y la computación de alta productividad (HTC) como medio para
    realizar investigaciones computacionalmente intensivas, haciendo énfasis en el potencial
    que tiene en las ciencias ambientales.

    \item Tesser R. y Borin E. en el 2022  presentan una revisión de la literatura
    donde se exponen los esfuerzos de combinar tecnologías como la virtualización basada
    en contenedores y la computación distribuida, específicamente en el área de la computación
    de alto rendimiento (HPC).

    %Trabajo relacionado transitivamente con la pregunta de investigación 2.
    \item Raj et al. en el 2020  presentan los antecedentes de HPC como objeto de estudio para estudiantes de pregrado en las universidades,
    resaltando las competencias y habilidades centrales que HPC requiere, así como su aplicación en distintos campos como la meteorología
    y la biología.

    \item Thain D., Tannenbaum T. y Livny M. en el 2005  describen el contexto histórico y coyuntural que dio
    origen al software HTCondor. Los autores describen en cómo HTCondor permite a los usuarios comunes
    acceder a grandes cantidades de poder computacional a través de un enfoque distribuido.
\end{itemize}

A pesar de que los trabajos anteriormente listados abordan aspectos relacionados con la computación
de alto rendimiento y el uso de HTCondor en diversas áreas, ninguno de ellos se enfoca específicamente
en los objetivos planteados en la presente investigación. En particular, la literatura revisada
no examina de manera sistemática los universos de HTCondor como objeto de estudio principal,
ni proporciona un análisis exhaustivo de sus aplicaciones y características en diferentes
contextos de investigación. Por consiguiente, se identifica una brecha de conocimiento que
justifica la realización del presente estudio de mapeo sistemático, el cual permitirá consolidar y
analizar de forma estructurada el estado actual del conocimiento sobre esta temática específica.


% --- REFERENCIAS ---
% Estilo de la bibliografía
    \bibliographystyle{IEEEtran}
% Archivo .bib donde se encuentran las referencias
%    \bibliography{resources/references.bib}
% -------------------


% ########################### FIN DE LAS SECCIONES ###########################

\end{document}